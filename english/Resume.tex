\documentclass [a4paper, 11pt]{article}
\usepackage{comfortaa}
%\renewcommand*\familydefault{\sfdefault} 
%\renewcommand*\rmdefault{ppl}
\renewcommand*\rmdefault{iwona}
\usepackage[T1]{fontenc}
\usepackage[none]{hyphenat}
\usepackage{color}
\usepackage[usenames,dvipsnames,svgnames,table]{xcolor}
\usepackage{graphicx}
\usepackage{tabularx}
\usepackage{sectsty}
\usepackage[a4paper, margin=2cm]{geometry}

\linespread{0.9}

\newcommand{\linedparagraph}[1]{\paragraph{#1}\mbox{}\\}

\renewcommand{\thefootnote}{\fnsymbol{footnote}}

\renewcommand{\arraystretch}{1.2}

\setlength{\tabcolsep}{.5cm}

\pagenumbering{gobble}

%Ejemplos de macros
% para definir macros inteligentes que anden bien, o sea que terminen con \xspace
\newcommand{\MACRO}[2]{\newcommand{#1}{#2\xspace}}
\newcommand{\MACROP}[3]{\newcommand{#1}[#2]{#3\xspace}}

\newcommand{\tabu}{\hspace*{0.7cm}}
\newcommand{\ctabu}{\hspace*{0.8cm}}
\newcommand{\htabu}{\hspace*{0.35cm}}

\begin{document}

\noindent \begin{tabularx}{\textwidth}{X r}
  \hspace{-15pt}\parbox[t]{15cm}{
  \hspace{-2pt}\textbf{\huge{Melanie Sclar}} \\
  Student in the Faculty of Exact and Natural Sciences \\
  of the University of Buenos Aires \\
  \emph{ \textcolor{gray}{melaniesclar@gmail.com} }
  } &

\parbox[t]{3.8cm}{ \emph{\textcolor{gray}{
  (+54)-911-6527-6507 \\
  (+54)-11-4581-2987 }}
  
}
\end{tabularx}

\sectionfont{\sectionrule{0pt}{0pt}{-.2cm}{1pt}}
\section* {Summary}
Computer Science and Mathematics student, experienced in problem solving and Algorithms, and looking forward to expand my knowledge and improve my skills in the field.
\section* {Education}

\begin{description}
  \item[University of Buenos Aires]{\hfill 2012-2016\footnote{Expected graduation date.} \\
	Currently pursuing Licenciatura en Ciencias de la Computaci\'on, \\
	equivalent to Bachelor of Computer Science +MSc. G.P.A. 9.60/10}

  \item[University of Buenos Aires]{\hfill 2012-2017\footnote{Expected graduation date.} \\
	Currently pursuing Licenciatura en Ciencias Matem\'aticas, \\
	equivalent to Bachelor of Mathematics +MSc. G.P.A. 9.00/10}
	
  \item[Superior School in Commerce ``Carlos Pellegrini'' - University of Buenos Aires] {\hfill 2007-2011 \\
	High School Degree with Accounting Specialization. G.P.A. 9.02/10}
\end{description}

\section* {Honors and Awards}

\subsection* {International Awards}

\begin{itemize} \itemsep.05cm
	\item[] Contestant in the 38th Annual World Finals of the ACM International Collegiate Programming Contest (ACM-ICPC) in Ekaterinburg, Russia (2014).
	\item[] Regional Champion in the South America/South Regional Contest of the ACM-ICPC (2013).
	\item[] Third place on the South America/South Regional Contest for the ACM-ICPC (2012).
	\item[] Honorable mention in the 24th Asian Pacific Mathematics Olympiad (2012).
	\item[] Bronze Medal on the 26th Iberoamerican Mathematical Olympiad (2011) in San Jose, Costa Rica.
	\item[] Silver Medal on the 21st Southamerican Mathematical Olympiad (2010) in S\~ao Paulo, Brazil. %-Olimpiada Matem\'atica del Cono Sur-


\end{itemize}

\subsection* {National Awards in Argentina}

\begin{itemize} \itemsep.05cm
	\item[] Third place in Argentine Programming Tournament (TAP), 2013.
	\item[] Second place in Argentine Programming Tournament (TAP), 2012.
	\item[] Third place in Argentine Tournament in Mathematics and Computing, 2011.
	\item[] National Champion in Argentine Olympiad in Mathematics (OMA), 2011.
	\item[] National Subchampion in Argentine Olympiad in Mathematics (OMA), 2008.
	
\end{itemize}

\newpage
\section* {Work Experience}

\begin{description}
  \item[Teacher Assistant at University of Buenos Aires] {\hfill March 2014 - ongoing\\
  Responsible of assisting students with their inquiries (involving class projects and exercises) and correcting exams. Currently teaching in Algorithms and Data Structures I.\\
}

\item[Coach for Mathematics Olympiads in:]
\item[\tabu Argentine Model School (EAM)] {\hfill March 2013 - ongoing}
\item[\tabu Martin Buber School] {\hfill March - December 2013}
\item[\tabu Superior School in Commerce ``Carlos Pellegrini'' (ESCCP)] { \hfill March - December 2012 \\

Helped students acquire problem solving techniques and mathematical concepts necessary in Mathematics Olympiads, and stimulated them to continue researching and solving problems on their own.
	}


\end{description}

\section* {Skills}

\begin {tabularx}{\textwidth}{>{\raggedright}>{\bfseries}X p{12cm}}
  Algorithms & Experienced due to my participation in informatics olympiads. \\
  Problem Solving & Passionate and highly experienced due to my participation in mathematical and computing olympiads.
\end {tabularx}

\subsection* {Programming Languages}

\begin {tabularx}{\textwidth}{>{\raggedright}>{\bfseries}X p{12cm}}
  C++ & Advanced knowledge, 5 years experience \\
  C & Intermediate knowledge, 5 years experience \\
  Haskell & Intermediate knowledge, 3 years experience \\
  Java & Basic knowledge, 1 year experience \\
  Smalltalk & Basic knowledge, 1 year experience \\
  \\
  \LaTeX & Intermediate knowledge, 3 years experience
\end {tabularx}

\subsection* {Spoken Languages}

\begin {tabularx}{\textwidth}{>{\raggedright}>{\bfseries}X p{12cm}}
  Spanish & Native Proficiency. \\
  English & Fluent Proficiency, able to use it professionally. \\
  French & Advanced Proficiency, able to hold a normal conversation.
\end {tabularx}

\end {document}
