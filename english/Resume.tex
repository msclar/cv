\documentclass [a4paper, 11pt]{article}
\usepackage{comfortaa}
%\renewcommand*\familydefault{\sfdefault} 
%\renewcommand*\rmdefault{ppl}
\renewcommand*\rmdefault{iwona}
\usepackage[T1]{fontenc}
\usepackage[none]{hyphenat}
\usepackage{color}
\usepackage{multicol}
\usepackage[usenames,dvipsnames,svgnames,table]{xcolor}
\usepackage{graphicx}
\usepackage{tabularx}
\usepackage{sectsty}
\usepackage[a4paper, margin=2cm]{geometry}

\linespread{0.9}

\newcommand{\linedparagraph}[1]{\paragraph{#1}\mbox{}\\}

\renewcommand{\thefootnote}{\fnsymbol{footnote}}

\renewcommand{\arraystretch}{1.2}

\setlength{\tabcolsep}{.5cm}

\pagenumbering{gobble}

%Ejemplos de macros
% para definir macros inteligentes que anden bien, o sea que terminen con \xspace
\newcommand{\MACRO}[2]{\newcommand{#1}{#2\xspace}}
\newcommand{\MACROP}[3]{\newcommand{#1}[#2]{#3\xspace}}

\newcommand{\tabu}{\hspace*{0.7cm}}
\newcommand{\ctabu}{\hspace*{0.8cm}}
\newcommand{\htabu}{\hspace*{0.35cm}}

\begin{document}

\noindent \begin{tabularx}{\textwidth}{X r}
  \hspace{-15pt}\parbox[t]{15cm}{
  \hspace{-2pt}\textbf{\huge{Melanie Sclar}} \\
  Computer Scientist, University of Buenos Aires \\
  \emph{ \textcolor{gray}{melaniesclar@gmail.com} }
  } &

\parbox[t]{3.8cm}{ \emph{\textcolor{gray}{
  (+54)-911-6527-6507 \\
  (+54)-11-4581-2987 }}
  
}
\end{tabularx}

\sectionfont{\sectionrule{0pt}{0pt}{-.2cm}{1pt}}
\section* {Summary}
Computer Scientist, Latin American champion in the 39th ACM-ICPC World Finals, Mathematics olympiad 
champion and former Facebook intern. Experienced in problem solving and complex algorithms and 
looking forward to expanding my knowledge and improving my skills in the field.

\section* {Work Experience}

\begin{description}
  \item[Lead Data Scientist at BrightSector Algorithms] {\hfill August 2018 - present\\
    Lead the team of Data Science \& ML, that focuses mainly on Named Entity Recognition of technical 
    attributes of Mercado Libre's publications. We are also in charge of classifying publications, clustering
    all of them that refer to the same product. Mercado Libre is the largest e-commerce site in Latin America.
  }

  \item[Data Scientist \& Machine Learning Engineer at BrightSector Algorithms] {\hfill January - July 2018\\
%    Research algorithms that recognizes and extracts several named entities from Mercado Libre's publications, such as an item's brand, dimensions, model, 
  }

  \item[Software Engineer Intern at Facebook Inc.] {\hfill January - April 2016\\
  Worked in the Video Encoding Team, on video stabilization algorithms. Researched various methods to 
  understand when stabilization is necessary and saving computing time while preserving the video's quality when it is not needed. 

  %Uniformed the formats of all videos on the site, which varied along the company's development. This heterogeneity of formats used to have various unwanted consequences on the system, which I fixed during the internship.
}

  \item[Software Engineer Intern at Facebook Inc.] {\hfill January - April 2015\\
Worked in the Public Content Ranking Team, developing algorithms to improve the quality of recommendations on videos and articles shown in newsfeed. These algorithms were based on sentiment analysis and trending topic detection. %These recommendations are other videos related to the original one and are shown to the user when he or she starts watching a video.
}

  \item[Teacher Assistant at University of Buenos Aires] {\hfill March 2014 - July 2017\\
  Responsible for assisting students with their inquiries (involving class projects and 
  exercises) and grading exams and projects. Taught mainly in Algorithms 
  and Data Structures III.
}

\item[Coach for Mathematics Olympiads] {\hfill March 2012 - October 2016 \\ 
\textbf{\small Argentine Model School (2013-2016), Martin Buber School (2013), Carlos Pellegrini School (2012)} \\
%\item[Coach for Mathematics Olympiads in:]
%\item[\tabu Argentine Model School (EAM)] {\hfill March 2013 - October 2016}
%\item[\tabu Martin Buber School] {\hfill March - December 2013}
%\item[\tabu Superior School in Commerce ``Carlos Pellegrini'' (ESCCP)] { \hfill March - December 2012 \\
%Helped students acquire problem solving techniques and mathematical concepts necessary in Mathematics Olympiads, and stimulated them to continue researching and solving problems on their own.
	}

\end{description}

\section* {Honors and Awards}

\subsection* {International Awards}
{\small
\begin{itemize} \itemsep.05cm
	\item[] Latin American Champion and First to Solve Problem F award in the 39th Annual World Finals of the ACM International Collegiate Programming Contest (ACM-ICPC) in Marrakech, Morocco (2015).
%	\item[] Contestant in the 38th Annual World Finals of the ACM International Collegiate Programming Contest (ACM-ICPC) in Ekaterinburg, Russia (2014).
	\item[] Champion in the South America/South Regional Contest of the ACM-ICPC (2013 and 2014).
	\item[] Third place in the South America/South Regional Contest for the ACM-ICPC (2012).
	\item[] Bronze Medal in the 26th Iberoamerican Mathematical Olympiad (2011) in San Jose, Costa Rica.
	\item[] Silver Medal in the 21st Southamerican Mathematical Olympiad (2010) in S\~ao Paulo, Brazil. 
\end{itemize}
}

\subsection* {National Awards in Argentina}

{\small
\begin{itemize} \itemsep.0cm
    \item[] First, third and second place in Argentine Programming Tournament (TAP) (2014, 2013 and 2012 respectively).
    \item[] Second place in the Interuniversity Argentine Mathematics Competition (CIMA) (2016).
	\item[] National Champion in Argentine Olympiad in Mathematics (OMA) (2011).
	\item[] National Subchampion in Argentine Olympiad in Mathematics (OMA) (2008).
\end{itemize}
}
\subsubsection* {National Awards in Machine Learning} \itemsep.05cm
{\small
\begin{itemize}
\item[] First place and second place in Almundo's ECI competition (2017 and 2018 respectively). ECI is the University of Buenos Aires's Invernal Computing School, that currently has an AI track.
\end{itemize}
}
\section* {Patents}
%Sclar M, Puntambekar A, Coward MH, Parsons-Keir, WD, inventors; Facebook Inc,
%assignee. $(1)$ Foreground detection for video stabilization and $(2)$ A neural network to 
%optimize video stabilization parameters. Attorney Docket Numbers 060406-8393.US00 
%and 060406-8394.US01, filed on June 30, 2016. Patent pending. \\

%(1) FOREGROUND DETECTION FOR VIDEO STABILIZATION (2) A NEURAL NETWORK TO OPTIMIZE VIDEO STABILIZATION PARAMETERS
% 060406.8393-8394.US01
% 060406-8393.US00

Sclar, M., Puntambekar, A., Coward, M.H. and Parsons-Keir, W.D., Facebook Inc, 2018. \textit{Foreground detection for video stabilization}. U.S. Patent Application 15/395,983. \\\\
Sclar, M., Puntambekar, A., Coward, M.H. and Parsons-Keir, W.D., Facebook Inc, 2018. \textit{Neural network to optimize video stabilization parameters}. U.S. Patent Application 15/396,025.

\section* {Volunteer experience}
\begin{description}
\item[Organizer and teacher of programming competitions' training events] {\hfill July 2014 - ongoing \\
  Taught in four two-week long events aimed at university students that 
  want to improve their skills in competitive programming, as well as 
  several events aimed at high school students.
}
\item[Organizer of programming and mathematical competitions] { \hfill July 2012 - ongoing \\
  Organizer of 15+ mathematical olympiads since 2012 and of the Argentinian Programming Tournament since 2016. Main organizer of 2019 South America Topcoder Open Regionals.
}
\end{description}

\section* {Education}
\begin{description}
  \item[University of Buenos Aires]{\hfill 2012-2018 \\
	Licenciada en Ciencias de la Computaci\'on, \\
	equivalent to Bachelor of Computer Science +MSc. G.P.A. 9.65/10 \\
    
    \textbf{Dissertation:} "Analysis and Prediction of Human Visual Search". This project consisted of creating an 
    algorithm involving machine learning and bayesian modelling to predict the most likely paths that 
    human gaze will make when searching for an object in an image. Publication pending.\\

    \textbf{Machine Learning courses} { \hfill \\
      Machine Learning, at Computer Science Department, UBA. \\ %(Classical supervised models (SVM, NN, EM, KNN)) \\
      Markov Processes applied to Deep Learning, at Applied Mathematics department, UBA. \\
    }

  \item[University of Buenos Aires]{\hfill 2012-ongoing \\ 
	Currently pursuing Licenciatura en Ciencias Matem\'aticas, \\
	equivalent to Bachelor of Mathematics +MSc. G.P.A. 9.00/10}
	
  \item[Superior School in Commerce ``Carlos Pellegrini'' - University of Buenos Aires] {\hfill 2007-2011 \\
	High School Degree with Accounting Specialization. G.P.A. 9.02/10}

}

\end{description}


\section* {Skills}

%\begin {tabularx}{\textwidth}{>{\raggedright}>{\bfseries}X p{11.9cm}}
%  Algorithms & Experienced due to my participation in informatics olympiads. \\
%  Problem Solving & Passionate and highly experienced due to my participation in mathematical and computing olympiads. \\
%  Machine Learning & Solid knowledge of the classic supervised and unsupervised models and with experience applying supervised models in structured data and images.
%\end {tabularx}

\begin{multicols}{2}
\subsection* {Programming Languages}
{\small
\begin {tabularx}{\textwidth}{>{\hsize=.12\hsize}>{\bfseries}X >{\hsize=.3\hsize}X}
  C / C++ & Upper intermediate, 7 years exp.\\
  Python & Upper intermediate, 4 years exp.\\
  Java & Upper intermediate, 2 years exp.\\
  Matlab & Intermediate, 3 years exp. \\
  PHP / Hack & Intermediate, 2 years exp.\\
  Haskell & Basic, 2 years exp.\\
  Bash & Basic, 2 years exp.\\
\end {tabularx}
}
\columnbreak
\subsection* {Spoken Languages}
{\small
\begin {tabularx}{\textwidth}{>{\hsize=.12\hsize}>{\bfseries}X >{\hsize=.3\hsize}X}
  Spanish & Native Proficiency \\
  English & Full professional proficiency \\
  Portuguese & Full professional proficiency \\
  French & Limited working proficiency
\end {tabularx}
}
\end{multicols}

\end {document}
